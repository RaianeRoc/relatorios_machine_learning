
\documentclass[        
    a4paper,          % Tamanho da folha A4
    12pt,             % Tamanho da fonte 12pt
    chapter=TITLE,    % Todos os capitulos devem ter caixa alta
    section=Title,    % Todas as secoes devem ter caixa alta somente na primeira letra
    subsection=Title, % Todas as subsecoes devem ter caixa alta somente na primeira letra
    oneside,          % Usada para impressao em apenas uma face do papel
    english,          % Hifenizacoes em ingles
    spanish,          % Hifenizacoes em espanhol
    brazil,           % Ultimo idioma eh o idioma padrao do documento
    fleqn             % Comente esta linha se quiser centralizar as equacoes. Comente também a linha 65 abaixo
]{abntex2}





% Importações de pacotes
\usepackage[utf8]{inputenc}                         % Acentuação direta
\usepackage[T1]{fontenc}                            % Codificação da fonte em 8 bits
\usepackage{graphicx}                               % Inserir figuras
\usepackage{amsfonts, amssymb, amsmath}             % Fonte e símbolos matemáticos
\usepackage{booktabs}                               % Comandos para tabelas
\usepackage{verbatim}                               % Texto é interpretado como escrito no documento
\usepackage{multirow, array}                        % Múltiplas linhas e colunas em tabelas
\usepackage{indentfirst}                            % Endenta o primeiro parágrafo de cada seção.
\usepackage{listings}                               % Utilizar código fonte no documento
\usepackage{xcolor}   
\usepackage{longtable}
\usepackage[table,xcdraw]{xcolor}
% \usepackage{microtype}                              % Para melhorias de justificação?
% \usepackage[portuguese,ruled,lined]{algorithm2e}    % Escrever algoritmos
% \usepackage{algorithmic}                            % Criar Algoritmos  
%\usepackage{float}                                 % Utilizado para criação de floats
\usepackage{amsgen}
\usepackage{lipsum}                                 % Usar a simulação de texto Lorem Ipsum
%\usepackage{titlesec}                              % Permite alterar os títulos do documento
\usepackage{tocloft}                                % Permite alterar a formatação do Sumário
\usepackage{etoolbox}                               % Usado para alterar a fonte da Section no Sumário
\usepackage[nogroupskip,nonumberlist]{glossaries}   % Permite fazer o glossário


\usepackage[font=singlespacing,format=plain,justification=justified,skip=0pt,singlelinecheck = false]{caption}            % Altera o comportamento da tag caption

\usepackage[alf, abnt-emphasize=bf, recuo=0cm, abnt-etal-cite=2, abnt-etal-list=0, abnt-etal-text=it]{abntex2cite}  % Citações padrão ABNT
%\usepackage[bottom]{footmisc}                      % Mantém as notas de rodapé sempre na mesma posição
%\usepackage{times}                                 % Usa a fonte Times
%%%%%%%%%%%%%%%%%%% AVISO %%%%%%%%%%%%%%%%%%%%%%%%%%%%%%%%%%%%%%%%
%descomete as duas linhas abaixo para alterar o texto de Times New Roman para Arial:

%\usepackage{helvet}
%\renewcommand{\familydefault}{\sfdefault}  % Usa a fonte Arial              
%%%%%%%%%%%%%%%%%%%%%%%%%%%%%%%%%%%%%%%%%%%%%%%%%%%%%%%%%%%%%%%%%%

\usepackage{mathptmx}         % Usa a fonte Times New Roman			%\usepackage{lmodern}         % Usa a fonte Latin Modern
%\usepackage{subfig}          % Posicionamento de figuras
%\usepackage{scalefnt}        % Permite redimensionar tamanho da fonte
\usepackage{color, colortbl} % Comandos de cores
%\usepackage{lscape}          % Permite páginas em modo "paisagem"
%\usepackage{ae, aecompl}     % Fontes de alta qualidade
%\usepackage{picinpar}        % Dispor imagens em parágrafos
\usepackage{latexsym}        % Símbolos matemáticos
%\usepackage{upgreek}         % Fonte letras gregas
\usepackage{appendix}         % Gerar o apêndice no final do documento
\usepackage{paracol}          % Criar parágrafos sem indentação
\usepackage{lib/ifcetex}	      % Biblioteca com as normas da ifce para trabalhos académicos
\usepackage{pdfpages}         % Incluir pdf no documento
\usepackage{amsmath}          % Usar equações matemáticas
% \usepackage{natbib}
\usepackage{graphicx}
\usepackage{subcaption}
\usepackage{booktabs}

\makeglossaries % Organiza e gera a lista de abreviaturas, símbolos e glossário
\makeindex      % Gera o Índice do documento         





\setlength{\mathindent}{0pt} %Complementa o alinhamento de equações para totalmente a esquerda.

%%%%%%%%%%%%%%%%%%%%%%%%%%%%%%%%%%%%%%%%%%%%%%%%%%%%%
%%                     ATENCAO                     %%
%%%%%%%%%%%%%%%%%%%%%%%%%%%%%%%%%%%%%%%%%%%%%%%%%%%%%
%  Qual e o nivel do trabalho academico que voce esta 
% escrevendo? Retire o simbolo "%" apenas de um dos 
% quatro topicos abaixo refente ao nível do seu traba
% -lho.

\trabalhoacademico{tccgraduacao}
%\trabalhoacademico{tccespecializacao}
% \trabalhoacademico{dissertacao}
%\trabalhoacademico{tese}

%%%%%%%%%%%%%%%%%%%%%%%%%%%%%%%%%%%%%%%%%%%%%%%%%%%%%

% Define se o trabalho e uma qualificacao
% Coloque 'nao' para versao final do trabalho

\ehqualificacao{nao}

% Remove as bordas vermelhas e verdes do PDF gerado
% Coloque 'sim' pare remover

\removerbordasdohyperlink{sim} 

% Adiciona a cor Azul a todos os hyperlinks

\cordohyperlink{nao}

%%%%%%%%%%%%%%%%%%%%%%%%%%%%%%%%%%%%%%%%%%%%%%%%%%%%%
%%         Informacao sobre a instituicao          %%
%%%%%%%%%%%%%%%%%%%%%%%%%%%%%%%%%%%%%%%%%%%%%%%%%%%%%

\ies{Instituto Federal de Educação, Ciência e Tecnologia do Ceará}
\iessigla{IFCE}
\campus{Campus Fortaleza}

%%%%%%%%%%%%%%%%%%%%%%%%%%%%%%%%%%%%%%%%%%%%%%%%%%%%%
%%        Informacao para TCC de Graduacao         %%
%%%%%%%%%%%%%%%%%%%%%%%%%%%%%%%%%%%%%%%%%%%%%%%%%%%%%

\graduacaoem{Aprendizado de Máquina}
% \habilitacao{licenciado} %ou bacharel/tecnólogo

% AVISO: Caso necessario alterar o texto de apresenta-
% cao da Especializacao, ir a pasta "lib", arquivo 
% "ifcetex.sty" na linha 502.


%%%%%%%%%%%%%%%%%%%%%%%%%%%%%%%%%%%%%%%%%%%%%%%%%%%%%
%%     Informacao para TCC de Especializacao       %%
%%%%%%%%%%%%%%%%%%%%%%%%%%%%%%%%%%%%%%%%%%%%%%%%%%%%%

% \especializacaoem{Yyyyyyyyy}

% AVISO: Caso necessario alterar o texto de apresenta-
% cao da Especializacao, ir a pasta "lib", arquivo 
% "ifcetex.sty" na linha 507.

%%%%%%%%%%%%%%%%%%%%%%%%%%%%%%%%%%%%%%%%%%%%%%%%%%%%%
%%         Informacao para Dissertacao             %%
%%%%%%%%%%%%%%%%%%%%%%%%%%%%%%%%%%%%%%%%%%%%%%%%%%%%%

% \programamestrado{Programa de Pós-Graduação em Xxxxxxx}
% \nomedomestrado{Mestrado Acadêmico em Xxxxxxx}
% \mestreem{Engenharia Xxxxxx}
% \areadeconcentracaomestrado{Engenharia Xxxxxx}

% AVISO: Caso necessario alterar o texto de apresenta-
% cao da dissertacao, ir a pasta "lib", arquivo 
% "ifcetex.sty" na linha 511.

%%%%%%%%%%%%%%%%%%%%%%%%%%%%%%%%%%%%%%%%%%%%%%%%%%%%%
%%               Informação para Tese              %%
%%%%%%%%%%%%%%%%%%%%%%%%%%%%%%%%%%%%%%%%%%%%%%%%%%%%%

% \programadoutorado{Programa de Pós-Graduação em Xxxxxx}
% \nomedodoutorado{Doutorado em Xxxxxxx}
% \doutorem{Engenharia Xxxxxx}
% \areadeconcentracaodoutorado{Engenharia Xxxxxxx}

% AVISO: Caso necessario alterar o texto de apresenta-
% cao da tese, ir a pasta "lib", arquivo "ifcetex.sty" 
% na linha 515.

%%%%%%%%%%%%%%%%%%%%%%%%%%%%%%%%%%%%%%%%%%%%%%%%%%%%%
%%      Informacoes relacionadas ao trabalho       %%
%%%%%%%%%%%%%%%%%%%%%%%%%%%%%%%%%%%%%%%%%%%%%%%%%%%%%

\autor{Raiane Rocha Reis}
\titulo{Relatório: Métodos de Classificação}
\data{2023}
\local{Fortaleza}

% Exemplo: \dataaprovacao{01 de Janeiro de 2012}
% \dataaprovacao{}

%%%%%%%%%%%%%%%%%%%%%%%%%%%%%%%%%%%%%%%%%%%%%%%%%%%%%
%%           Informação sobre o Orientador         %%
%%%%%%%%%%%%%%%%%%%%%%%%%%%%%%%%%%%%%%%%%%%%%%%%%%%%%

% \orientador{Dr. Xxxxxx Xxxxx Xxxxxx}
\orientadories{Instituto Federal de Educação, Ciência e Tecnologia do Ceará (IFCE)}
% \orientadorcentro{Centro de Ciências e Tecnologia (CCT)}
% \orientadorfeminino{nao} % Coloque 'sim' se for do sexo feminino

%%%%%%%%%%%%%%%%%%%%%%%%%%%%%%%%%%%%%%%%%%%%%%%%%%%%%
%%          Informação sobre o Coorientador        %%
%%%%%%%%%%%%%%%%%%%%%%%%%%%%%%%%%%%%%%%%%%%%%%%%%%%%%

% Deixe o nome do coorientador em branco para remover do documento

\coorientador{}
\coorientadories{}
% \coorientadorcentro{Centro do Coorientador (SIGLA)}
\coorientadorfeminino{nao} % Coloque 'sim' se for do sexo feminino

%%%%%%%%%%%%%%%%%%%%%%%%%%%%%%%%%%%%%%%%%%%%%%%%%%%%%
%%              Informação sobre a banca           %%
%%%%%%%%%%%%%%%%%%%%%%%%%%%%%%%%%%%%%%%%%%%%%%%%%%%%%

% Atenção! Deixe em branco o nome do membro da banca para remover da folha de aprovacao

% Exemplo de uso:
% \membrodabancadois{Prof. Dr. Fulano de Tal}
% \membrodabancadoisies{Universidade Federal do Ceará - ifce}


% \membrodabancadois{Dr. Xxxxxxx Xxxxxx Xxxxxxx}
% \membrodabancadoiscentro{Centro de Física Teórica e Computacional}
% \membrodabancadoisies{Universidade de Lisboa (ULisboa)}
% \membrodabancatres{Dr. Xxxxxxx Xxxxxx Xxxxxxx}
% \membrodabancatrescentro{Xxxxxxx Xxxxxx Xxxxxxx}
% \membrodabancatresies{Universidade do Membro da Banca Três (SIGLA)}
% \membrodabancaquatro{Prof. Dr. Xxxxxxx Xxxxxx Xxxxxxx}
% \membrodabancaquatrocentro{Centro de Ciências e Tecnologia (CCT)}
% \membrodabancaquatroies{Universidade do Membro da Banca Quatro (SIGLA)}
% \membrodabancacinco{Prof. Dr. Xxxxxxx Xxxxxx Xxxxxxx}
% \membrodabancacincocentro{Teste}
% \membrodabancacincoies{Universidade do Membro da Banca Cinco (SIGLA)}
% \membrodabancaseis{Prof. Dr. Xxxxxxx Xxxxxx Xxxxxxx}
% \membrodabancaseiscentro{}
% \membrodabancaseisies{Universidade do Membro da Banca Seis (SIGLA)}

\begin{document}	

	% Elementos pré-textuais
	%Se não for usar algum dos elementos, deixe-o comentado
	\imprimircapa
	% \imprimirfolhaderosto{}
	% \imprimirfichacatalografica{1-pre-textuais/ficha-catalografica}
	% \imprimirerrata{1-pre-textuais/errata}
	% \imprimirfolhadeaprovacao
	% \imprimirdedicatoria{1-pre-textuais/dedicatoria}
	% \imprimiragradecimentos{1-pre-textuais/agradecimentos}
	% \imprimirepigrafe{1-pre-textuais/epigrafe}
	% \imprimirresumo{1-pre-textuais/resumo}
	% \imprimirabstract{1-pre-textuais/abstract}
	% \renewcommand*\listfigurename{Lista de Figuras} %Se você comentar esta linha o título da lista fica: LISTA DE ILUSTRAÇÕES
	% \imprimirlistadeilustracoes
	% \imprimirlistadetabelas
	% \imprimirlistadequadros
	% \imprimirlistadealgoritmos
	% \imprimirlistadecodigosfonte
	% \imprimirlistadeabreviaturasesiglas
	% \imprimirlistadesimbolos{1-pre-textuais/lista-de-simbolos}   
	\imprimirsumario
	
	\setcounter{table}{0}% Deixe este comando antes da primeira tabela.
	
	%Elementos textuais
	\textual
	\input{2-textuais/1-introducao}
	% 

\chapter{Referencial Teórico}
\label{chap:referencial-teorico}

\section{MLP (Perceptron Multicamadas)}

Como é mostrado na Figura \ref{fig:subfigura1}, uma Rede Neural Multicamadas (MLP – \textit{MultiLayer Perceptron})
é formada por um conjunto de neurônios, também conhecidos como
Perceptrons. Uma MLP consiste em uma camada de entrada, juntamente com uma ou mais camadas ocultas.
No processo de treinamento, é empregada uma técnica chamada retropropagação (\textit{backpropagation}), que
ocorre em duas fases distintas: a propagação para frente (\textit{forward}) e a retropropagação propriamente
dita (\textit{backward}), assim como ilustra a Figura \ref{fig:subfigura2}. Durante a propagação para frente, os dados são passados pela rede, camada por camada,
permitindo que as saídas da rede sejam calculadas. Em seguida, durante a fase de retropropagação, os
erros entre as saídas previstas e os valores reais são calculados e propagados de volta através da rede,
ajustando os pesos das conexões para minimizar esses erros. Esse processo iterativo é fundamental para o
treinamento eficaz de uma MLP, permitindo que ela aprenda e se adapte \cite{su12114776}.  

\begin{figure}[h]
    \centering
    \caption{Estrutura e atividade de uma rede MLP, imagens de \cite{su12114776}.}
    \begin{subfigure}{0.4\textwidth}
      \centering
      \includegraphics[width=\linewidth]{figuras/MLP/rede_MLP.png}
      \caption{Camadas de uma MLP.}
      \label{fig:subfigura1}
    \end{subfigure}
    \hspace{5mm}
    \begin{subfigure}{0.45\textwidth}
      \centering
      \includegraphics[width=\linewidth]{figuras/MLP/atividade_MLP.png}
      \caption{\textit{Forward} e \textit{backward} em uma rede MLP.}
      \label{fig:subfigura2}
    \end{subfigure}
    \label{fig:subfiguras}
  \end{figure}

\section{Naive Bayes}

Naive Bayes um método de classificação e tem como base o teorema de Bayes. Este método
gera uma tabela de probabilidades utilizadas para classificar os dados, as features dos 
dados são analisadas de forma independente, por isso o nome \textit{Naive}, que significa
ingênuo.

\begin{itemize}
    \item Teorema de Bayes:
    \begin{equation}
        P(y|x) = P(y)P(x|y)/P(x)
        \label{eq:mass_energy_equation}
    \end{equation}

\end{itemize}
% \item 



\section{SVM com kernel RBF}



\section{SVM com kernel Polinomial}

\section{SVM com kernel Linear}

	% \chapter{Justificativa}
\label{chap:justificativa}


	% \chapter{Objetivos}
\label{chap:objetivos}

\section{Objetivo Geral}

\begin{itemize}
    \item Objetivo Geral.
\end{itemize}

\section{Objetivos Específicos}

\begin{itemize}
    \item Objetivo específico 1;
    \item Objetivo específico 2.
\end{itemize}
	\chapter{Metodologia}

\section{SVR (SVM para Regressão)}

\section{Regressão Simples}

\section{Regressão Múltipla}

	\chapter{Resultados e Discussão}
\label{chap:resultados}

\section{Seção 1}



\section{Seção 2}



\section{Seção 3}
	\input{2-textuais/7-conclusao}
	
	%Elementos pós-textuais	
	\bibliography{3-pos-textuais/referencias}
%	\imprimirglossario	
	% \imprimirapendices
		% Adicione aqui os apendices do seu trabalho
		% \input{3-pos-textuais/apendices/apendice-a}
		% \input{3-pos-textuais/apendices/apendice-b}
		% \input{3-pos-textuais/apendices/apendice-c}
		% \input{3-pos-textuais/apendices/apendice-d}
	% \imprimiranexos
		% Adicione aqui os anexos do seu trabalho
	% 	\input{3-pos-textuais/anexos/anexo-a}
	% 	\input{3-pos-textuais/anexos/anexo-b}		
	% \imprimirindice


\end{document}